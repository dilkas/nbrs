\documentclass[runningheads]{llncs}
\usepackage{amsmath}
\usepackage{amsfonts}
\usepackage{color}
\usepackage{graphicx}
\usepackage{listings}
% If you use the hyperref package, please uncomment the following line
% to display URLs in blue roman font according to Springer's eBook style:
% \renewcommand\UrlFont{\color{blue}\rmfamily}

\DeclareMathOperator{\Dist}{Dist}
\DeclareMathOperator{\ctrl}{ctrl}
\DeclareMathOperator{\prnt}{prnt}
\DeclareMathOperator{\link}{link}

\lstdefinelanguage{Big}{
  sensitive=true,
  morekeywords={fun, brs, end, sbrs, pbrs, nbrs, begin, init, atomic, preds,
    rules, action, big, ctrl, float, int, react},
  comment=[l]{\#}
}

\begin{document}
\title{Nondeterministic Bigraphical Reactive Systems for Markov Decision
  Processes\thanks{Supported by organization x.}}
%\titlerunning{Abbreviated paper title}
% If the paper title is too long for the running head, you can set
% an abbreviated paper title here
\author{Paulius Dilkas\orcidID{0000-1111-2222-3333} \and
  Michele Sevegnani\orcidID{1111-2222-3333-4444}}
\authorrunning{P. Dilkas \& M. Sevegnani}
% First names are abbreviated in the running head.
% If there are more than two authors, 'et al.' is used.
%
\institute{University of Glasgow, Glasgow, UK}
\maketitle

\begin{abstract}
The abstract should briefly summarize the contents of the paper in
150--250 words.

\keywords{First keyword  \and Second keyword \and Another keyword.}
\end{abstract}

\section{Introduction}

\begin{definition}[\cite{DBLP:conf/nfm/GiaquintaHIMN18}]
  For any finite set $X$, let $\Dist(X)$ denote the set of discrete probability
  distributions over $X$. A \emph{Markov Decision Process} is a tuple $ (S,
  \overline{s}, A, P, L)$, where: $S$ is a finite set of states and
  $\overline{s} \in S$ is the initial state; $A$ is a finite set of
  \emph{actions}; $P : S \times A \to \Dist(S)$ is a (partial) probabilistic
  transition function, mapping state-action pairs to probability distributions
  over $S$; $L : S \to 2^P$ is a labelling with atomic propositions.
\end{definition}

\begin{definition}
  A \emph{reward structure} for an MDP $(S, \overline{s}, A, P, L)$ is a pair
  $(\rho, \iota)$, where $\rho : S \to \mathbb{R}$ is the \emph{state reward
    function}, and $\iota : S \times A \to \mathbb{R}$ is the \emph{transition
    reward function}.
\end{definition}

% TODO rephrase
\begin{definition}
  A \emph{bigraph} is a tuple $(V, E, \ctrl, \prnt, \link) : \langle k, X
  \rangle \to \langle m, Y \rangle$, where $V$ is a set of nodes, $E$ is a set
  of edges, $\ctrl$ is the \emph{control map} that assigns controls to nodes,
  $\prnt$ is the \emph{parent map} that defines the nesting of nodes, and
  $\link$ is the \emph{link map} that defines the link structure.

  The notation $\langle k, X \rangle \to \langle m, Y \rangle$ indicates that
  the bigraph has $k$ \emph{holes} (sites) and a set of inner names $X$ and $m$
  \emph{regions}, with a set of \emph{outer names} $Y$. These are respectively
  known as the \emph{inner} and \emph{outer} interfaces of the bigraph.
\end{definition}

Reaction rule definition \cite{DBLP:journals/corr/abs-1111-0086}

Stochastic bigraphs \cite{DBLP:journals/entcs/KrivineMT08}

PhD thesis \cite{DBLP:phd/ethos/Sevegnani12}

Define NBRS

Lemma: any MDP can be expressed as an NBRS

Changes from PBRS: each reaction rule annotated with an action name
(probabilities normalised over each action separately) and an integer
reward/cost, predicates get a reward/cost, too.

\section{Example}

\begin{figure}
  \includegraphics{models/example_ts.pdf}
  \caption{...}
  \label{example_ts}
\end{figure}

\lstinputlisting[caption=asdf, language=Big]{models/example.big}

\begin{example}
  Consider an MDP $(S, \overline{s}, A, P, L)$, where...
  Look at Figure~\ref{example_ts}.
\end{example}

\section{Exporting to PRISM}

Transitions, state rewards, transition rewards

\section{Case study in autonomous agents}

\bibliographystyle{splncs04}
\bibliography{bibliography}
\end{document}
